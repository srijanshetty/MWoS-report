\documentclass[11pt,oneside]{book}
\pagestyle{plain}
\usepackage{graphics}
\usepackage{makeidx}
\usepackage{nomencl}

\textheight=235mm
\textwidth=145mm
\topmargin=0mm
\headheight=0mm
\headsep=0mm

% These must be changed for double sided output !

\oddsidemargin=15mm
\evensidemargin=15mm

\makeindex
\makeglossary
\begin{document}

% set line spacing to 1.5B
\baselineskip = 17pt


\title{Multifactor Authentication and Session Support in OpenVPN}
\author{Report submitted in accordance with the requirements of \\
the Indian Institute of Technology, Kanpur \\
by \\
Harshwardhan Sharma (),\\
Shivanshu Agarwal (),\\
Srijan R. Shetty (11727)}
\date{November 2014}
\maketitle
\frontmatter

\chapter{Abstract}
Mozilla uses OpenVPN with MFA via deferred C plugins and pythons scripts.
However, there are several caveats that require non-plugin based modifications,
such as One Time Passwords (OTP) client input and session tracking.
The goal of this project has been to research and provide a first class user experience
to the end user when using MFA with OpenVPN; including the support for session resumption
and backwards compatibility with older versions of OpenVPN.

\tableofcontents

% Contents not put in table of contents by default so add it separately

\addcontentsline{toc}{chapter}{Contents}

\listoffigures


% List of figures  not put in table of contents by default so add it separately

\addcontentsline{toc}{chapter}{List of Figures}


\chapter{Acknowledgement}

The author wishes to thank the University of
Liverpool Computing Services Department for the development of this
\LaTeX \ thesis template.

\printglossary

\addcontentsline{toc}{chapter}{Nomenclature}





\mainmatter

\chapter{How to use the Thesis Template}
\section{Introduction}

\LaTeX \ is a document publishing system which can produce high quality
output particulary when complicated formulae need to be reproduced in print.
Because of this it is widely used by mathematicians. \LaTeX \ is essentially
a collection of \TeX \ macros which is a typesetting language in its own right.
\TeX \ is quite difficult to learn but provides much finer control over typsetting
then \LaTeX. Most users will find that \LaTeX \ meets most of their
needs and will therefore have little need of \TeX. \LaTeX \ is available on
both the MWS and Sun UNIX services with the MWS
version having the added advantage of including a graphical user interface called
TeXShell.

There are no courses provided by CSD in \LaTeX \ however, information is available
on line. The TeXShell/MikTeX guide can be found at:
\\
\\
http://www.liv.ac.uk/CSD/acuk\_html/414.dir/414.doc
\\
\\
and there is also an FAQ at:
\\
\\
http://www.liv.ac.uk/csd/software/ps/faqtex.htm
\\
\\
The Sun UNIX \LaTeX \ guide is at:
\\
\\
http://www.liv.ac.uk/CSD/acuk\_html/452.dir/452.pdf
\\
\\
If you are new to \LaTeX, then probably the best place to start is with
Leslie Lamport's very readable guide, {\em ``LaTeX - A Document Preparation
System''} \cite{latex} . If you need to delve into \TeX, then {\em ``The TeXbook''} \cite{texbook} by
Donald E. Knuth is essential reading. Knuth, a leading light in
computer science, designed the \TeX \ language in order to produce
his seminal work {\em ``The Art of Computer Programming''} \cite{artbook} so this really
is the definitive guide to the subject.

To save users from having to type in a many low level
\TeX \ commands, style files can be used to define additional
macros which can be used in addition to standard ones from
\LaTeX. Many journal publishers provide style files to ensure
that all submissions conform to a standard `house' format. It may
be worth checking on the Internet to see if style files are available
to suit your particular discipline.

\section{Running \LaTeX}
This section descrines how to take the thesis template file
({\bf thesis.tex}) and convert it into a PostScript or Portable
Document Format (PDF) file which can be printed on a high quality
laser printer. Both PostScript and PDF files can be previewed
prior to printing using either Ghostview or Adobe's Acorobat
viewer. There are some common pitfalls and misconceptions associated with
with Ghostview and PostScript, and neophyte\nomenclature{{\bf Neophyte}}{A beginner or novice.} users are strongly
advised to take a look at the ``FAQ for Ghostview and PostScript"
at:
\\
\\
http://www.liv.ac.uk/csd/software/latex/faqtex.htm
\\
\\
Both \LaTeX \ and \TeX \ convert the original ``source" file
into a DVI file (by convention having the {\bf .dvi} extension)
before this is converted to PostScript. DVI is short
for DeVice Independent format, meaning that it works
independently of the printer format (usually PostScript)
which is used to produced the hard copy version. Despite
what the name suggests, DVI files are seldom useful outside
of the \LaTeX \ environment.

Historically speaking, PostScript has been widely used to
publish \LaTeX \ derived articles
on the World Wide Web but, since the arrival of PDF, this should be
stricly discouraged. PostScript is a format designed for use with
laser printers and many modern web browsers have difficulty displaying
files which contain it. On the other hand most browsers will display
PDF files and PDF is becoming an increasing popular standard for sharing
textual information
\footnote{
You might wish to consider how best to save your thesis in electronic format
for posterity. Saving the \LaTeX \ original as a plain text file and
the formatted version as a PDF file
onto
ordinary CD-R is probably the safest bet at the moment, although bear in mind
that the computer graveyard is full of ``standards" of depressingly short longevity.
Fortunately, \LaTeX \ is likely to be far more future proof than this week's version
of Microsoft Word.
}.

\subsection{\LaTeX \ on the Sun UNIX Service}

To ``compile" the template file ({\bf thesis.tex}) into a DIV file, the
following command is used:

\begin{verbatim}
$ latex thesis.tex
\end{verbatim}

Alternatively, in the unlikely circumstances that the input file contains pure
\TeX \ , use:

\begin{verbatim}
$ tex pure_tex_file.tex
\end{verbatim}

Assuming that this does not throw up any errors, you can then convert the
DVI file to PostScript format using:

\begin{verbatim}
$ dvips thesis
\end{verbatim}

You can then preview the output to ensure that what you get from the
printer is actually what you wanted using Ghostview as
follows:

\begin{verbatim}
$ ghostview thesis.ps &
\end{verbatim}


Finally, to print the output to a laser printer, use:

\begin{verbatim}
$ lpr -Pps thesis.ps
\end{verbatim}


By default, this will go to the central printer in Brownlow Hill;
the {\bf -P} option can be used to direct the output elsewhere. It is
straightforward to convert the PostScript version to PDF:

\begin{verbatim}
$ ps2pdf thesis.ps thesis.pdf
\end{verbatim}

and infact it is possible to convert \LaTeX \ documents to many other formats, including
HTML. See the TeX FAQ for details.


\subsection{\LaTeX \ on the Managed Windows Service}
\LaTeX \ on the Managed Windows Service comes in two parts namely:
a group of programs run from the DOS command line called MikTeX
and a graphical windows interface called TeXShell. If neither
of these appear to be installed on your MWS PC click

\begin{verbatim}
Start | Install | Office | MikTeX 2.1
\end{verbatim}

The TeXShell interface is by far the easiest to work with although
MikTeX provides additional programs not accessible through
the standard TeXShell interface
\footnote
{Although they can be included --- see the FAQ for details.}.
To compile the template file click on
\begin{verbatim}
File | Open
\end{verbatim}
in TeXShell and browse to where you have saved {\bf thesis.tex}.
Then click on the LaTeX button (use the TeX button for plain \TeX \ files).

To convert the DVI output
to PostScript, click on the Dvips button. You can then
preview the output prior to printing by clicking the
Ghostview button. The document can then be printed
by clicking File $|$ Print inside Ghostview.

\section{Some examples of \LaTeX \ text formatting}

\subsection {Plain Text}
Type your text in free-format; lines can be as long
or as short
as you wish.
        You can indent         or space out
        your input
            text in
                any way you like to highlight the structure
        of your manuscript and make it easier to edit.
LaTeX fills lines and adjusts spacing between words to produce an
aesthetically pleasing result.

Completely blank lines in the input file break your text into
paragraphs.
To change the font for a single character, word, or set of words,
enclose the word and the font changing command within braces,
{\em like this}.
A font changing command not enclosed in braces, like the change to \bf
bold here, keeps that change in effect until the end of the document or
until countermanded by another font switch, like this change back to
\rm roman.

\subsection {Displayed Text}
Use the ``quote'' and  ``quotation'' environments for typesetting quoted
material or any other text that should be slightly indented and set off
from the normal text.
\begin{quotation}
The quote and quotation environments are similar, but use different
settings for paragraph indentation and spacing.

\em When in doubt, consult the manual.
\end{quotation}


\begin{enumerate}
\item
The ``enumerate'' environment numbers the list elements, like this.

Items in a list can contain multiple paragraphs.
These paragraphs are appropriately spaced and indented according to their
position in the list.
   \begin{itemize}
   \item The ``itemize'' environment sets off list items with ``bullets'',
like this.  Finally, the ``description'' environment lets you put your own
      \begin{description}
      \item[A] label on each item, like this ``A''.
      \item[If the label is long,] the first line of the item text will
be spaced over to the right as needed.
      \end{description}
   \item Of course, lists can be nested, each type up to at least four levels.
One type of list can be nested within another type.
      \begin{itemize}
      \item Nested lists of the same type will change style of numbering
or ``bullets'' as needed.
      \end{itemize}
   \end{itemize}
\item Don't forget to close off all list environments with the
appropriate \verb+\end{...}+ command.
Indenting \verb+\begin{...}+, \verb+\item+, and \verb+\end{...}+
commands in the input document according to their nesting level can help
clarify the structure.
\end{enumerate}

Here is a very simple table showing data lined up in columns.
Notice that the table is in a ``center'' environment to display
it properly.
The title is created simply as another paragraph in the center environment,
rather than as part of the table itself.
\begin{center}
Numbers of Computers Network, By Type.

\begin{tabular}{lr}
Macintosh&175\\
DOS/Windows PC&60\\
UNIX Workstation or server&110\\
\end{tabular}
\end{center}

Here is a more complicated table that has been boxed up, with a multi-column
header and paragraph entries set in one of the columns.
\begin{center}
\begin{tabular}{|l|c|p{3.5in}|}
\hline
\multicolumn{3}{|c|}{Places to Go Backpacking}\\ \hline
Name&Driving Time&Notes\\
&(hours)&\\ \hline
Big Basin&1.5&Very nice overnight to Berry Creek Falls from
either Headquarters or ocean side.\\ \hline
Sunol&1&Technicolor green in the spring.  Watch out for the cows.\\ \hline
Henry Coe&1.5&Large wilderness nearby suitable for multi-day treks.\\ \hline
\end{tabular}
\end{center}

\subsection {Mathematical Equations}
Simple equations, like $x^y$ or $x_n = \sqrt{a + b}$ can be typeset right
in the text line by enclosing them in a pair of single dollar sign symbols.

A more complicated equation should be typeset in {\em displayed math\/} mode,
like this:
\[
p(x)=\lim_{N \rightarrow \infty, \Delta x \rightarrow 0}\sum_{j=1}^{K}(f_j/\Delta x)
\]
The ``equation'' environment displays your equations, and automatically
numbers them consecutively within your document, like this:
\begin{equation}
\mu_m=\sum_{i=0}^m(-1)^i{m \choose i}\mu_1^i\mu_{m-i}^{\prime} \; \; {\rm where} \; \; {m \choose i}={m! \over i!(m-i)!}.
\label{eq:summation1}
\end{equation}

\pagebreak
\noindent{\Large\bf Here is the input file that produced these formatting examples:}
\begin{verbatim}

\subsection {Plain Text}
Type your text in free-format; lines can be as long
or as short
as you wish.
        You can indent         or space out
        your input
            text in
                any way you like to highlight the structure
        of your manuscript and make it easier to edit.
LaTeX fills lines and adjusts spacing between words to produce an
aesthetically pleasing result.

Completely blank lines in the input file break your text into
paragraphs.
To change the font for a single character, word, or set of words,
enclose the word and the font changing command within braces,
{\em like this}.
A font changing command not enclosed in braces, like the change to \bf
bold here, keeps that change in effect until the end of the document or
until countermanded by another font switch, like this change back to
\rm roman.

\subsection {Displayed Text}
Use the ``quote'' and  ``quotation'' environments for typesetting quoted
material or any other text that should be slightly indented and set off
from the normal text.
\begin{quotation}
The quote and quotation environments are similar, but use different
settings for paragraph indentation and spacing.

\em When in doubt, consult the manual.
\end{quotation}


\begin{enumerate}
\item
The ``enumerate'' environment numbers the list elements, like this.

Items in a list can contain multiple paragraphs.
These paragraphs are appropriately spaced and indented according to their
position in the list.
   \begin{itemize}
   \item The ``itemize'' environment sets off list items with ``bullets'',
like this.  Finally, the ``description'' environment lets you put your own
      \begin{description}
      \item[A] label on each item, like this ``A''.
      \item[If the label is long,] the first line of the item text will
be spaced over to the right as needed.
      \end{description}
   \item Of course, lists can be nested, each type up to at least four levels.
One type of list can be nested within another type.
      \begin{itemize}
      \item Nested lists of the same type will change style of numbering
or ``bullets'' as needed.
      \end{itemize}
   \end{itemize}
\item Don't forget to close off all list environments with the
appropriate \verb+\end{...}+ command.
Indenting \verb+\begin{...}+, \verb+\item+, and \verb+\end{...}+
commands in the input document according to their nesting level can help
clarify the structure.
\end{enumerate}

Here is a very simple table showing data lined up in columns.
Notice that the table is in a ``center'' environment to display
it properly.
The title is created simply as another paragraph in the center environment,
rather than as part of the table itself.
\begin{center}
Numbers of Computers Network, By Type.

\begin{tabular}{lr}
Macintosh&175\\
DOS/Windows PC&60\\
UNIX Workstation or server&110\\
\end{tabular}
\end{center}

Here is a more complicated table that has been boxed up, with a multi-column
header and paragraph entries set in one of the columns.
\begin{center}
\begin{tabular}{|l|c|p{3.5in}|}
\hline
\multicolumn{3}{|c|}{Places to Go Backpacking}\\ \hline
Name&Driving Time&Notes\\
&(hours)&\\ \hline
Big Basin&1.5&Very nice overnight to Berry Creek Falls from
either Headquarters or ocean side.\\ \hline
Sunol&1&Technicolor green in the spring.  Watch out for the cows.\\ \hline
Henry Coe&1.5&Large wilderness nearby suitable for multi-day treks.\\ \hline
\end{tabular}

\subsection {Mathematical Equations}
Simple equations, like $x^y$ or $x_n = \sqrt{a + b}$ can be typeset right
in the text line by enclosing them in a pair of single dollar sign symbols.

A more complicated equation should be typeset in {\em displayed math\/} mode,
like this:
\[
p(x)=\lim_{N \rightarrow \infty, \Delta x \rightarrow 0}\sum_{j=1}^{K}(f_j/\Delta x)
\]
The ``equation'' environment displays your equations, and automatically
numbers them consecutively within your document, like this:
\begin{equation}
\mu_m=\sum_{i=0}^m(-1)^i{m \choose i}\mu_1^i\mu_{m-i}^{\prime} \; \; {\rm where} \; \; {m \choose i}={m! \over i!(m-i)!}.
\end{equation}

\end{verbatim}
\section{Including figures}
\LaTeX \ can be used to include figures in documents provided that
they are in {\em Encapsulated} PostScript (EPS) format (ordinary PostScript
is not suitable). The process of including figures can be quite
fiddly and if you need to include a large number of figures it
may be quicker and easier to use a desktop publishing package.
Generally line drawings work better than pictures (i.e. raster
or bitmapped graphics) as the latter take up considerable disk space
and take long periods of time to print. There are different methods
which can be used to include figures. One is to place the
\verb+\usepackage{graphics}+ command at the beginning of the file
and then use the
\verb+\includegraphics+ command to include
the EPS file as has been employed here:

\section{Keep reference information upto date }

In a document as large and as complicated as a thesis the ability to
cross-reference information contained both within the thesis itself
and in external references quickly and easily is obviously essential.
Clearly the author does not want to have to change the reference
numbering every time a reference is added or removed. Fortunately
\LaTeX \ can ensure that the numbering is automatically kept
consistent. References can be cross-references to other information
in the thesis in the form of equations, figures, theorems etc
or they can be references to entries in the bibliography. \LaTeX \ can
also ensure that the tables of contents and figures reflect the current
page numbering. It can also produce a glossary and index automatically. All
of the topics are described in more detail below.

\subsection{Cross-references}
To create
a cross-reference, the
\verb+\label+ command is used to provide the information
with a key which \LaTeX \ can then use to generate a cross-reference
number for using the
\verb+\ref+ command. For example, this should
contain the equation number for the second equation displayed above (\ref{eq:summation1}).
The \verb+\label+ command was used earlier to provide a key for the equation
and the \verb+\ref+ command used here to generate the corresponding equation number.
You can extend this idea to figures (e.g. the ULGrid logo: Figure \ref{fig:logo})
or hypotheses, theorems, lemmas, corollaries \ldots infact just about
anything you may want to cross-reference. Note that when you first run
{\bf latex/dvips} on the document in which a reference has been created, the
reference number may not appear correctly. This is because \LaTeX \ cannot
possibly know the number of a reference occuring later in the document to
where it is was cited. To do this, it must scan through the original TeX
file twice and, indeed, the way to ensure that cross-references are upto date
and consistent is just to run {\bf latex} twice before running {\bf dvips}
\footnote{Those from a computing background may see that this is analagous
to how a two--pass compiler resolves forward references.}.

\subsection{Creating a table of contents}
A table of contents can be generated in \LaTeX \ simply by including
the command \verb+\tableofcontents+. As with all references, it may be
necessary to run {\bf latex} twice in order to generate the numbering
correctly. This has been used in this thesis template.

\subsection{Creating a table of figures}
A table of figures can similarly be generated in \LaTeX \ by simply including
the command \verb+\listoffigures+. As with all references, it may be
necessary to run {\bf latex} twice in order to produce the correct numbering.
This has been used in this thesis template.

\subsection{Creating a glossary}
A glossary can be created in a similar manner to an index. The
\verb+\makeglossary+ is placed in the preamble to document and
\verb+\nomenclature+ used instead of \verb+\index+ to add
glossary entries. The \verb+\printglossary+ command is to indicate
where the glossary is to appear. Creating the index files for a glossary
is slightly more complicated. For this template file the following command
is used:

\begin{verbatim}
$ makeindex -s nomencl -o thesis.gls thesis.glo
\end{verbatim}

The term ``glossary"\nomenclature{{\bf Glossary}}{A list of often difficult or specialized words with their
definitions usually placed in technical literature.} has been included as an example.

\subsection{Creating an index}
Although indices \index{index} \index{indices|see{index}} are difficuly to produce and are
fairly rare in theses, \LaTeX \ can take
out much of
the tedious work needed in creating one. To create
an index\index{index!\LaTeX \ commands} you will need to include the commands \verb+\usepackage{makeindex}+
and \verb+\makeindex+ at the start of the document. The command \verb+\printindex+
is used to indicate where you want the index to appear (usually right at the
end of the document). To index a particular term place the command \verb+\index+
as close as possible to it in the text. You can find a number of references to
this section in the index at the end of this template. Creating the index
requires a little additional work when running \LaTeX. After running the
{\bf latex} command to compile the document, then run\index{index!using makeindex}

\begin{verbatim}
$ makeindex thesis
\end{verbatim}

and run {\bf latex} again twice in order to include the index in the output
({\bf .dvi}) file. Further details can be found in Lamport's book \cite{latex}.


\subsection{Bibliography and citation}
A citation is a cross-reference to another publication, such as a journal
article called the {\em source}. With \LaTeX \ , the citation is produced
by a \verb+\cite+ command having the citation key as its argument. \LaTeX
uses a separate program called BIBTeX to produce the source list for the
program. You can find details of how to do this in Lamport's book \cite{latex}.
Sources are placed in a separate bibliographic database file ({\bf .bib} file).
The bibliography can be formatted using a variety of predefined styles or you may
be able to find a style sheet that more accurately reflects the format
used in your particular disciple. With BIBTeX it is first necessary to
run {\bf latex} on the main document, then {\bf bibtex} on the {\bf .bib} file
then {\bf latex} twice on the main document again. It may be worth keeping a record
of all references you come across in the {\bf .bib} file for future use as it
can be used in many different publications (e.g. journal papers).

\section{Footnotes}
Footnotes can be added quite easily by using the \verb+\footnote+ command
such as in this example \footnote{There are several footnotes in this template.}.
The University rules state that footnotes should be labelled consecutively
throughout the document. This is what \LaTeX \ does by default.

\section{Splitting the input}

\LaTeX \ uses sophisticated algorithms to decide when to split lines
and how much space to insert between words, essentially mimicking the, now
bygone, art of manual type setting (composition). Such attention
to detail ensures that the quality of \LaTeX \ output is unsurpassed but it
also comes at a cost. Compiling a \LaTeX \ document, particularly one as large
as a thesis, takes a considerable time which isn't incurred by WYSIWIG
\nomenclature{{\bf WYSIWIG}}{An acronymn for {\em `What you see is what you get'} often
applied to word processors although many in the \TeX \ community believe
that {\em `What you see is all you've got'} is closer to the truth.}
word processors such as Microsoft Word (which consequently produce far inferior
quality output). Since only small changes are generally made to the document
at a time it makes sense to split the document into smaller parts and compile
these individually using the {\bf latex} command only when they need to be.
Splitting the input also means
that the author can edit much smaller individual files rather than
an entire tome\nomenclature{{\bf Tome}}{A book, especially a large or scholarly one.} each time.

The \LaTeX command
\verb+\include+ can be used to gather the sections together in a ``root"
document and this has been used to split Chapter 2 of this template into
a separate file ({\bf chapter2.tex}).  For a thesis it, may
well be worth working on an individual chapter at a time and then combining
them together at the end. The root document will need to be compiled
and any indexes created to ensure the correct numbering before printing
out the final hard copy version.

% \include{chapter2}


\appendix
\chapter{Example Appendix}
Appendices are usually labelled with letters separate to ordinary chapters.


\bibliographystyle{plain}
\bibliography{thesis}
\addcontentsline{toc}{chapter}{Bibliography}

\printindex
\addcontentsline{toc}{chapter}{Index}

\begin{verbatim}










\end{verbatim}
\end{document}
